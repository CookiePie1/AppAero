\documentclass[12pt]{article}
\usepackage[latin1]{inputenc}
\usepackage[margin=1in]{geometry}
\usepackage{amsmath}
\usepackage{amsfonts}
\usepackage{amssymb}
\usepackage{amsthm}
\setcounter{section}{-1}
% cover page inspired by Tarang S.

\usepackage{tocloft}
\renewcommand\cftsecfont{\normalfont}
\renewcommand\cftsecpagefont{\normalfont}
\renewcommand{\cftsecleader}{\cftdotfill{\cftsecdotsep}}
\renewcommand\cftsecdotsep{\cftdot}
\renewcommand\cftsubsecdotsep{\cftdot}

\usepackage{fancyhdr}
\pagestyle{fancy}
\fancyhf{}
\rhead{J. Yang}
\lhead{Applied Aeronautics}
\rfoot{Page \thepage}
\setlength\parindent{0pt}
\begin{document}
	\begin{titlepage}
		\begin{center}
			\vspace*{1cm}
			\Huge
			\textbf{Applied Aeronautics}\\
			\vspace{0.5cm}
			\LARGE
			Independent Study Course Plan\\
			\vspace{1.5cm}
			\textbf{John Yang}\\
			\vfill
			\vspace{0.8cm}
			\Large
			Dr. Mesut B. \c{C}akir\\
			South Brunswick High School\\
			January 2020
		\end{center}
	\end{titlepage}
\tableofcontents\newpage

\section{Course Overview}
The independent study course in Applied Aeronautics serves as a practical and hands-on introduction to the fields of aerodynamics, airframe design, and aircraft systems. 

\subsection{Overall Learning Objectives}
By the end of the course, the student aims to:
\begin{itemize}
	\item Gain an intuitive understanding of fluids, aerodynamics, aircraft design, and aircraft performance
	\item Be able to accurately and effectively analyze, simulate, and/or debrief a given situation
	\item Apply their knowledge in the subject using working physical models
\end{itemize}
\subsection{Learning Methods}
This is not a traditional course of study, thus traditional learning methods should not be expected. The student should be a self-motivated and self-driven learner who can work independently with or without teacher supervision. For each topic, the student is expected to find appropriate resources to guide their learning. The student should then take notes on the topic until they have reached a self-determined level of understanding. The student will then apply their knowledge in one or multiple projects. 
\subsection{Methods of Assessment}
Again, because this is not a traditional course, traditional assessments and tests will not be used. Instead, the student will use the following flow:
\begin{enumerate}
	\item The student will self-reflect on their own progress and learning. Ask the following questions:
	\begin{enumerate}
		\item What are you trying to learn or gain from this topic?
		\item Have you achieved that?
		\item Could you explain this topic coherently to someone with no prior knowledge of the subject? Would they understand?
	\end{enumerate}
	The student may choose whether they would like to internalize, verbalize, or record their responses in writing. 
	\item If the student passes their own test, they may move onto the next step. If not, they should return to the material, look over their notes again, and consider what they are missing or what they do not yet understand. They should then fill in any gaps in their knowledge and administer the self-assessment again. 
	
\end{enumerate}
\subsection{Resources and materials}

\section{Quarter One Class Plans}
\section{Quarter Two Class Plans}
\section{Quarter Three Class Plans}
\section{Quarter Four Class Plans}
\end{document}