\documentclass{article}
\title{Notes Section 4} % title
\author{John Yang}
\usepackage{amsmath}
\usepackage[margin=1in, letterpaper]{geometry}
\usepackage{outlines}
\setcounter{section}{3} % chapter number minus 1
\usepackage{mathtools}
\DeclarePairedDelimiter\set\{\}
\usepackage{hyperref}
\hypersetup{
	colorlinks=true,
	linkcolor=blue,
	filecolor=magenta,      
	urlcolor=cyan,
}
\usepackage{tocloft}
\renewcommand\cftsecfont{\normalfont}
\renewcommand\cftsecpagefont{\normalfont}
\renewcommand{\cftsecleader}{\cftdotfill{\cftsecdotsep}}
\renewcommand\cftsecdotsep{\cftdot}
\renewcommand\cftsubsecdotsep{\cftdot}

\begin{document}
    \maketitle
    \tableofcontents
    \section{Cities and 20th Century Architecture}
    \small Twentieth-Century Architecture, Dennis P. Doordan \normalsize 
    \subsection{Responses to the Modern City}
    \begin{outline}
        \1 The modern city vs. traditional city: new skyscrapers, cinemas, department stores, etc. become part of the city center rather than churches, palaces, small shops, etc. 
        \1 Ebeneezer Howard's Garden City (1898):
            \2 Howard wanted to connect the cultural and educational benefits of urban life with the health benefits of country life. 
            \2 The plan consisted of a central garden, around which consecutive rings protruded. It was seen as somewhat practical and not far fetched or eccentric like current ideas of a utopia. 
        \1 Daniel Burnham and Edward Bennett: The City Beautiful and the Chicago plan 
            \2 1909 Chicago Plan was comissioned by business elites in the city. 
            \2 The City Beautiful movement sought to combat negative impacts of uncoordinated and uncontrolled growth in cities; improve traffic; protect public health; and provide opportunities for cultural development \& recreation. 
            \2 1909 Chicago Plan: tried to add diagonal streets to the grid system of Chicago. Had a civic center to the west of the Loop and cultural/educational center to the east. Large parks by the lakefront. 
            \2 The 1909 Chicago Plan was not directly implemented, but it did influence later public policy and planning decisions. Keep in mind that this plan was commisioned by the rich elites of Chicago. 
        \1 Tony Garnier's Industrial City 
            \2 Urban plan and ideas based on politics and power: a single administrative agency controls land, food, utilities, supplies, etc. Suggest a socialist system. 
            \2 Designs primarily use reinforced concrete
        \1 Antonio Sant'Elia's Futurist City 
            \2 Futurism celebrates technological advancements rather than traditional ideas of beauty or art 
            \2 People criticise avant-garde ideas, which seemed to attack traditional values without really offering any constructive benefit or viable alternative. 
            \2 Basic and high-level ideas exist in these plans, but lack the details to make implementation even remotely possible. Many small but important details are not worked out. 
    \end{outline}
    \subsection{Problems with urban planning}
    \begin{outline}
        \1 Most cities are built building-by-building; a cohesive plan that is implemented all at once is impossible unless widespread demolition is conducted. 
        \1 Unfortunately, most large-scale urban planning exercises end up being theoretical, or a few components selected from them. Some projects can be implemented on a small scale (see: IIT, Chicago, Mies van der Rohe)
    \end{outline}
\end{document}
