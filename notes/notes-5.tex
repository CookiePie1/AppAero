\documentclass{article}
\title{Notes Section 5} % title
\author{John Yang}
\usepackage{amsmath}
\usepackage[margin=1in, letterpaper]{geometry}
\usepackage{outlines}
\setcounter{section}{4} % chapter number minus 1
\usepackage{mathtools}
\DeclarePairedDelimiter\set\{\}
\usepackage{hyperref}
\hypersetup{
	colorlinks=true,
	linkcolor=blue,
	filecolor=magenta,      
	urlcolor=cyan,
}
\usepackage{tocloft}
\renewcommand\cftsecfont{\normalfont}
\renewcommand\cftsecpagefont{\normalfont}
\renewcommand{\cftsecleader}{\cftdotfill{\cftsecdotsep}}
\renewcommand\cftsecdotsep{\cftdot}
\renewcommand\cftsubsecdotsep{\cftdot}

\begin{document}
    \maketitle
    \tableofcontents
    \section{Wright Stuff}
    Note: this is a running document. \\
    \small \href{https://freeflight.org/wp-content/uploads/2016/02/2009ScienceOlympiadManual.pdf}{Chuck Markos - SO plane guide} \normalsize 
    \subsection{Building}
    \begin{outline}
        \1 Materials:
            \2 Balsa wood:
                \3 \href{https://specializedbalsa.com/}{Specialized Balsa}
            \2 Carbon fiber:
                \3 
            \2 Carbon rods: 
                \3 Dragonplate
            \2 Covering
                \3 Mylar ultrafilm
                \3 Tissue paper
            \2 Adhesive
                \3 Glues: Super glue, Bob Smith CA, Duco cement
                \3 Epxoy: Great Planes Epoxy
                \3 Tape: Packing, masking - for construction purposes
            \2 Rubber
                \3 Widths: This year's rules - prob aroung .05 g/in
                \3 Mass is limited to 1.5g
                \3 Length: no shorter than 1x hook-to-hook distance, no longer than 2x hook-to-hook. 
            \2 Other Materials
                \3 Tissue tubes
                \3 
    \end{outline}
    \subsection{Trimming}
    \begin{outline}
        \1 
    \end{outline}
\end{document}