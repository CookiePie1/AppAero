\documentclass{article}
\title{Notes Section 10} % title
\author{John Yang}
\usepackage{amsmath}
\usepackage[margin=1in, letterpaper]{geometry}
\usepackage{outlines}
\setcounter{section}{0} % chapter number minus 1
\usepackage{mathtools}
\DeclarePairedDelimiter\set\{\}
\usepackage{hyperref}
\hypersetup{
	colorlinks=true,
	linkcolor=blue,
	filecolor=magenta,      
	urlcolor=cyan,
}
\usepackage{tocloft}
\renewcommand\cftsecfont{\normalfont}
\renewcommand\cftsecpagefont{\normalfont}
\renewcommand{\cftsecleader}{\cftdotfill{\cftsecdotsep}}
\renewcommand\cftsecdotsep{\cftdot}
\renewcommand\cftsubsecdotsep{\cftdot}

\begin{document}
    \maketitle
    \tableofcontents
    \section{The rise of the suburbs}
    \subsection{Economic factors that influenced suburban growth}
    \begin{outline}
        \1 After the Great Depression, people lost their homes en masse. Government loan programs allowed mortages to be taken out over 20-30 years (the now standard) instead of what was then the standard of 5 years. This made housing much more accessible to the average worker and their family. Housing demand grew massively, as did home building. 
        \1 The Federal Housing Authority (FHA) insured some mortages by protecting the lenders even if the loanee defaulted. Multiple other factors also influenced the rapid growth of suburbs, including the building of Interstate Highways, the GI Bill, the general productive state of the economy after World War II, etc. 
        \1 William Levitt designed the "typical" suburban community by buying a huge expanse of land, cutting it up into lots, and building numerous houses on the land. Economies of scale allowed each house to be built cheaply. 
    \end{outline}
    \subsection{Suburban form and design}
    \begin{outline}
        \1 
    \end{outline}
\end{document}