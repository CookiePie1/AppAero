\documentclass{article}
\title{Aero Sect. 1 Notes} % title
\author{John Yang}
\usepackage{amsmath}
\usepackage{amssymb}
\usepackage[margin=1in, letterpaper]{geometry}
\usepackage{outlines}
\setcounter{section}{0} % chapter number minus 1
\usepackage{mathtools}
\DeclarePairedDelimiter\set\{\}
\usepackage{hyperref}
\hypersetup{
	colorlinks=true,
	linkcolor=blue,
	filecolor=magenta,      
	urlcolor=cyan,
}
\usepackage{tocloft}
\renewcommand\cftsecfont{\normalfont}
\renewcommand\cftsecpagefont{\normalfont}
\renewcommand{\cftsecleader}{\cftdotfill{\cftsecdotsep}}
\renewcommand\cftsecdotsep{\cftdot}
\renewcommand\cftsubsecdotsep{\cftdot}

\begin{document}
    \maketitle
    \tableofcontents
    \section{Basic Aerodynamic Principles}
    \subsection{Vocabulary}
    \begin{outline}
        \1 Pressure: limiting form of the force per unit area: \[p=\lim\left(\dfrac{dF}{dA}\right)\]
            \2 Pressure is a point property; it can have different values within the fluid. 
        \1 Density: \[\rho=\lim\dfrac{dm}{dv}\qquad dv\to0\]
        \1 Aerodynamics - fluids in motion is key; we use streamlines to represent them. 
        \1 Center of pressure - the location where the resultant of a distributed load effectively acts on an aerodynamic body. 

    \end{outline}
    \subsection{Engineering applications of lift and drag coefficients}
    \begin{outline}
        \1 Engineers are concerned with the coefficients of lift and drag and how they are changed inflight, unintentionally or intentionally. 
        \1 Certain devices can be used to intentionally change these coefficients, like flaps, slats, spoilers, etc. 
        \1 Lift and drag coefficients are constantly changing; they can depend on the airspeed, angle of attack, and many other factors. Thus, engineers must analyze and design aircraft based on empirical data. 

    \end{outline}
\end{document}