\documentclass{article}
\title{Sect. 1 Notes} % title
\author{John Yang}
\usepackage{amsmath}
\usepackage{amssymb}
\usepackage[margin=1in, letterpaper]{geometry}
\usepackage{outlines}
\setcounter{section}{0} % chapter number minus 1
\usepackage{mathtools}
\DeclarePairedDelimiter\set\{\}
\usepackage{hyperref}
\hypersetup{
	colorlinks=true,
	linkcolor=blue,
	filecolor=magenta,      
	urlcolor=cyan,
}
\usepackage{tocloft}
\renewcommand\cftsecfont{\normalfont}
\renewcommand\cftsecpagefont{\normalfont}
\renewcommand{\cftsecleader}{\cftdotfill{\cftsecdotsep}}
\renewcommand\cftsecdotsep{\cftdot}
\renewcommand\cftsubsecdotsep{\cftdot}

\begin{document}
    \maketitle
    \tableofcontents
    \section{Basic Aerodynamic Principles}
    \subsection{Vocabulary}
    \begin{outline}
        \1 Pressure: limiting form of the force per unit area: \[p=\lim\left(\dfrac{dF}{dA}\right)\]
            \2 Pressure is a point property; it can have different values within the fluid. 
        \1 Density: \[\rho=\lim\dfrac{dm}{dv}\qquad dv\to0\]
        \1 Aerodynamics - fluids in motion is key; we use streamlines to represent them. 
        \1 Center of pressure - the location where the resultant of a distributed load effectively acts on an aerodynamic body. 

    \end{outline}
    \subsection{Engineering applications of lift and drag coefficients}
    \begin{outline}
        \1 Engineers are concerned with the coefficients of lift and drag and how they are changed inflight, unintentionally or intentionally. 
        \1 Certain devices can be used to intentionally change these coefficients, like flaps, slats, spoilers, etc. 
        \1 Lift and drag coefficients are constantly changing; they can depend on the airspeed, angle of attack, and many other factors. Thus, engineers must analyze and design aircraft based on empirical data. 

    \end{outline}
    \section{Principles of Urbanism}
        \subsection{What makes a city a city? What is urbanism?}
        \small{\href{https://www.urban.org/urban-wire/what-defines-city}{The Urban Institute: Neighborhoods, Cities, and Metros}}\\
        \small{\href{https://www.yorku.ca/anderson/Intro%20Urban%20Studies/Unit1/what_is_urban.htm}{What Is Urban? York University}}
        \normalsize
        \begin{outline}
            \1 Cities vs metro areas: the city area is typically much smaller than the greater metro area. The city limits are defined by land boundaries (New York consists of the five boroughs, yet the greater NYC metro area includes and extends farther - to long island, near upstate, some of north and central NJ, etc. This is why the term ``Chicagoland" exists). 
            \1 Definitions and characteristics of cities change depending on how you choose to define them. 
            \1 An ``urban" area is really not carefully or precisely defined. You can view an urban area from a few perspectives:
                \2 Population denisty: urban areas have many people living in high-density situations; that is, many people/families/individuals live in a small land footprint. This means that living spaces can be smaller and space is often built vertically. 
                \2 Landscape: Urban areas tend to have buildings packed closely together, contain high amounts of impermeable materials such as asphalt, concrete, class, steel, etc. 
                \2 Function: Urban areas are places where people live urban lifestyles, connect with people through urban infrastructure, participate in an urban economy, etc. Notice how we are still not defining ``urban". 
            \1 Obviously urban areas exist, but to what extent have they blended into their surroundings? In the case of the NY/NJ area, the suburbs and urban areas are so close together and there are no real boundaries between them. Interconnectedness through physical transportation and internet interactions allow people to take part in some of these ``urban" activities in locations that are not necessarly ``urban". 
        \end{outline}
        \subsection{How do we see and observe a city?}
        \small 11.001J - Part 1
        \normalsize 
        \begin{outline}
            \1 
        \end{outline}
        \subsection{The ``Ideal City'' Paintings: Early Ideas of a Utopia by Design}
        \small{\href{https://www.archiobjects.org/the-ideal-city-in-three-renaissance-paintings/}{The “Ideal City” in three Renaissance paintings}}\\
        \small{\href{https://www.nytimes.com/2012/05/09/arts/09iht-conway09.html}{NY Times - If a City Were Perfect, What Would It Look Like?}}
        \normalsize 
        \begin{outline}
            \1 The Ideal Cities show a more artistic/architectural perspective of Urban design through the lens of the Renaissance. 
            \1 This becomes markedly more abstract than any scientific study of urbanism, but it is still helpful to capture the ideas represented by it. 
            \1 The Ideal City (Urbino Panel):
                \2 A description of the painting: The painting has an extreme sense of symmetry and detail. It is drawn in one-point perspective. The painting shows the viewer looking upon a plaza in front of a circular building. The ground appears to be made of white, finished masonry with blue accents. To the sides of the round building are a series of different buildings that extend into the distance, each a different color. At the street level, there are arches that visually support the buldings. In the foreground, there are two column bases \textit{that do not extend higher than the waist level}. The sky has three distinct shades of blue, with the darkest in front, and becoming white in the distance. There are a few grey clouds in the sky. There are no people displayed in the painting whatsoever. 
                \2 Some unanswered questions about what is going on within the painting: 
                    \3 What is the purpose/point/function of the two unfinished columns? They are not unfinished per se; there is no exposed sign of construction; yet they are not columns. They may represent obstacles, blockages, or tools for the soft delineation of space, but why \textit{this}?
            \1 The Ideal City of Baltimore (Baltimore Panel):
                \2 Description: The Baltimore Panel shares some similarities but many differences to the Urbino Panel. Here, there is a plaza, again made of stone with grey accents and small grassy areas. In the center is a small fountain. There are four narrow columns in the plaza. The most notable detail of this panel is that the plaza is recessed into the ground; there are several steps down into the plaza. Behind the plaza are three buildings: In the center, a grand arch, to the left, an ampitheatere, and to the right, a building resemblant of the Baptisery of Florence. In the nearer right, there is a white building with window details, but no arches. To the left, there is a building with mildly ornamented windows and square arches. The viewpoint of the painting is similar to the Urbino panel, and the sky blue in a similar way. 
                \2 Note that art historians believe that the humans in the image were added at a later date, after the original work was completed. 
            \1 The Ideal City of Berlin (Berlin Panel):
                \2 Description: This panel departs from the others, though it encapsulates similar ideas to Baltimore and Urbino. The perspective is the same. However, the point of view is from underneath a partially open-air structure. The lighting is notably darker from the lack of direct sunlight. There are four columns holding up the roof in the foreground. To the left and right are two arches that are also part of the sturcture. The stone-clad street extends outward with a grid-like pattern. In the far distance is a body of water with a few boats. To the sides are buildings with varying facades and details but the same overall style. A building to the right has a archway that seems to be an extension of the street; it can be used both as an entryway to the building but also as a throughway to wherever someone is going. The building the perspective is under is towards the back of an intersection or square; the beginnings of the first buildings in the perspective are offset to the back, signifying the presence of a square or street that runs perpendicular to the visible one. 
            \1 What makes these cities ideal? 
                \2 Soft: perfect symmetry, statliness, cleanliness, proportion. 
                \2 Tangible: all have a central public area, signifying the importance of the convergence of people and social interaction; the buildings are inviting to the inhabitants of the city. 
                \2 \textit{Even though none of the images contain people, it is evident that the purpose of the city's design, structure, and physical aspects aim primarily to serve people; the citizens come first. }
        \end{outline}
\end{document}