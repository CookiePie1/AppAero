\documentclass{article}
\title{Notes Section 6} % title
\author{John Yang}
\usepackage{amsmath}
\usepackage[margin=1in, letterpaper]{geometry}
\usepackage{outlines}
\setcounter{section}{5} % chapter number minus 1
\usepackage{mathtools}
\DeclarePairedDelimiter\set\{\}
\usepackage{hyperref}
\hypersetup{
	colorlinks=true,
	linkcolor=blue,
	filecolor=magenta,      
	urlcolor=cyan,
}
\usepackage{tocloft}
\renewcommand\cftsecfont{\normalfont}
\renewcommand\cftsecpagefont{\normalfont}
\renewcommand{\cftsecleader}{\cftdotfill{\cftsecdotsep}}
\renewcommand\cftsecdotsep{\cftdot}
\renewcommand\cftsubsecdotsep{\cftdot}

\begin{document}
    \maketitle
    \tableofcontents
    \section{11.550x - Human Factors in Mobility Design}
    \subsection{Psychology in transportation - Cognitive Biases}
    \begin{outline}
        \1 Overconfidence biases: When people overestimate their skills relative to others, causing them to engage in risk behaviors which leads to injuries and death. 
        \1 Framing: People make decisions based on how information is presented to them; the decisions are based on the connotation of each choice. 
        \1 Loss aversion: People prefer to avoid losses than to acquire incomparable gains. In other words, people hate losing things. 
            \2 In transportation: activities like walking and biking cause you to lose \$0 a month, while driving causes you to lose, say \$10 a month. If we leverage people to see the non-losses of alternative transport, we improve urban mobility. 
        \1 Hindsight Bias: When people overestimate their own ability to have predicted the outcome that they wouldn't have been able to predict before the event occurred. 
            \2 In transportation: people believe that driving to work will be faster than other modes, but they could not have predicted the time it would have taken to get to work in either scenario. 
        \1 Confirmation bias: People seek evidence/data/information that supports an opinion that they already have
        \1 Self-serving bias: The tendency to attribute positive outcomes to personal skill and negative outcomes to luck. 
        \1 Narrative fallacy: We naturally like to hear stories, and as a result, our opinions are influenced more readily by them. 
        \1 Anchoring bias: Uses preexisting data as a reference point for everything else that comes after. For example, you might think that a certain car is cheaper, period, because you saw an expensive car first. 
        \1 Herd mentality: Tendency for crowds to follow what everyone else is doing, whether it is a celebrity or just a large group of people. 
            \2 In transportation, if a bunch of people decide to start using public transit instead of cars, it would cause others to follow the bandwagon. 
        \1 Representativeness heuristic: Our minds try to find similarities between things and as a result, we tend to stereotype. It can cause people to falsely equate different things. 
    \end{outline}
    \subsection{Changing behavior}
    \begin{outline}
        \1 Phases of behavior change (transtheoretical model)
            \2 Precontemplation $\to$ Contemplation $\to$ Determination $\to$ Action $\to$ Relapse $\to$ Maintenance $\to$ Precontemplation (and so on)
            \2 A person may exit or reenter at any point within the cycle. 
                \3 Precontemplation: people don't intend to take any action in the forseeable future (6 months); they are unaware of their negative behavior. 
                \3 Contemplation: people start to realize that their behavior is problematic and intend to take action soon (within 6 months). 
                \3 Determination/preparation: people are ready to make a change within the next month. They take small steps towards the positive outcome. 
                \3 Action: people have recently changed their behavior within the past 6 months and intend to continue. 
                \3 Maintenance: people have changed their behavior for over 6 months, and intend to continue. They work on preventing \textbf{relapse} into earlier stages. 
                \3 Termination: people have no desire to return to the problematic behavior and are completely sure they will not relapse. This stage is rarely reached. 
        \1 Behavioral interventions
            \2 Ways to influence people to change their behavior
            \2 Example 1: a study had participants log their transit habits for a week; at the end of the week, a reseracher discussed the results with them and suggested specific improvements that could be made. 
            \2 Example 2: commuter challenge - people commute to work using different modes of transportation in a mini experiment and see which one is the fastest. It could be a personal project or just something that someone does and publishes it to the media. Usually, the bicycle wins and using a helicopter doesn't win. 
            \2 Example 3: social sanctions: in Bogota, the government had people dress up as traffic cones and playfully harrass people who parked illegally, causing embarassement as a deterrent for illegal parking. The key here is that it is done seemingly jokingly and not in a hostile way, but it still achieves its intended result. 
            \2 Other strategies: you could make things into a game with incentives
            \2 Tactical urbanism: visually change a space, like a parking spot or new bike lane (with things like paint, furniture, plants, etc.), to signify that it is now serving a different purpose. 
    \end{outline}
    \subsection{Equity}
    \begin{outline}
        \1 In transportation planning, certain vulnerable groups are often overlooked and therefore are not included in transit opportunities/accessibility. These groups include: children, women, elderly people, disabled people, and BIPOC communities. 
        \1 People who are disadvantaged by lacking transit accessibility miss out on economic opportunities and are subject to a perpetuation of poverty. 
        \1 Transit poverty: 
            \2 Availability - this not only pertains to where and when transit is available, but also to the modes of transit, length of trips, accessibility (in a reduced physical mobility sense), and suitability for children. 
            \2 Accessibility - is transit actually reaching the places where people need to go? Do the destinations cater to all people, or only to working men? All infrastructure planning requires transit and accessibility planning as well. 
            \2 Affordability - if riders need to take several different modes of transit to get where they need, they often have to pay several fares, which discourages them from using transit. Even in households with cars, women often don't get the chance to actually use them because the relative monetary value of the men's work is higher. However, women still need to do whatever they need to do and thus rely on transit. 
            \2 Time poverty - women often need to travel to multiple different locations. Speed of public transit is often lacking, and as a result, they have less time available to carry out their tasks because most of it is spent on transit. 
            \2 Transit adequacy - women often face sexual harrassment and discrimination, or even assault on public transit. How can we make transit safer for women, and thus make it more available and desireable to them?
        \1 Steps toward equity 
            \2 Recall that a pivotal moment during the American Civil Rights movement was with Rosa Parks, who refused to move to the back of the bus. The resulting bus boycotts saw people walking to their destinations rather than taking the bus. 
            \2 making transit more equitable not only impacts disadvantaged communities, but its economic benefits also improve the rest of society. 
            \2 Make transit more equitable by: improving access to disadvantaged communities by adding new modes, new stations, or reduced fares. 
            \2 The first step in improving equity is to understand the communities you are serving - the demographics, socioeconomic conditions, etc. 
            \2 Transit equity is not gender neutral - you MUST consider the safety and accessibility of women and men when planning tranist because they have different requirements and needs than others. 

    \end{outline}
    \subsection{Health and environment}
    \begin{outline}
        \1 Climate change - current car usage is having devastating effects on the environment, and a shift towards public transit prominence has been shown to decrease the rate of climate effects. Public tranist reduces the amonut of emmisions per capita, and living in urban environments reduces emissions because everything you need is located closer together. 
        \1 Public tranist also reduces the relative amount of raw materials needed per passenger, as the vehicles have higher capacity and are rarely sitting idle as cars do. 
        \1 Electric cars - they have many of the same problems that cars do. Additionally, the source for the electricity ends up being fossil fuel anyway, depending on how the electricity is obtained. 
        \1 Cars are unsafe for people because of the relative lack of training that drivers undergo, and the unpredictability of accidents, other drivers, etc. 
        \1 Current tranist modes are also medically unhealthy - there is less walking, biking, or innate activity in lifestyles; rather, people are forced to use their own initiative to exercise which is not as effective as a necessity of exercise. 
        
    \end{outline}
\end{document}