\documentclass{article}
\title{Notes Section 6} % title
\author{John Yang}
\usepackage{amsmath}
\usepackage[margin=1in, letterpaper]{geometry}
\usepackage{outlines}
\setcounter{section}{5} % chapter number minus 1
\usepackage{mathtools}
\DeclarePairedDelimiter\set\{\}
\usepackage{hyperref}
\hypersetup{
	colorlinks=true,
	linkcolor=blue,
	filecolor=magenta,      
	urlcolor=cyan,
}
\usepackage{tocloft}
\renewcommand\cftsecfont{\normalfont}
\renewcommand\cftsecpagefont{\normalfont}
\renewcommand{\cftsecleader}{\cftdotfill{\cftsecdotsep}}
\renewcommand\cftsecdotsep{\cftdot}
\renewcommand\cftsubsecdotsep{\cftdot}

\begin{document}
    \maketitle
    \tableofcontents
    \section{11.550x - Human Factors in Mobility Design}
    \subsection{Psychology in transportation - Cognitive Biases}
    \begin{outline}
        \1 Overconfidence biases: When people overestimate their skills relative to others, causing them to engage in risk behaviors which leads to injuries and death. 
        \1 Framing: People make decisions based on how information is presented to them; the decisions are based on the connotation of each choice. 
        \1 Loss aversion: People prefer to avoid losses than to acquire incomparable gains. In other words, people hate losing things. 
            \2 In transportation: activities like walking and biking cause you to lose \$0 a month, while driving causes you to lose, say \$10 a month. If we leverage people to see the non-losses of alternative transport, we improve urban mobility. 
        \1 Hindsight Bias: When people overestimate their own ability to have predicted the outcome that they wouldn't have been able to predict before the event occurred. 
            \2 In transportation: people believe that driving to work will be faster than other modes, but they could not have predicted the time it would have taken to get to work in either scenario. 
        \1 Confirmation bias: People seek evidence/data/information that supports an opinion that they already have
        \1 Self-serving bias: The tendency to attribute positive outcomes to personal skill and negative outcomes to luck. 
        \1 Narrative fallacy: We naturally like to hear stories, and as a result, our opinions are influenced more readily by them. 
        \1 Anchoring bias: Uses preexisting data as a reference point for everything else that comes after. For example, you might think that a certain car is cheaper, period, because you saw an expensive car first. 
        \1 Herd mentality: Tendency for crowds to follow what everyone else is doing, whether it is a celebrity or just a large group of people. 
            \2 In transportation, if a bunch of people decide to start using public transit instead of cars, it would cause others to follow the bandwagon. 
        \1 Representativeness heuristic: Our minds try to find similarities between things and as a result, we tend to stereotype. It can cause people to falsely equate different things. 
    \end{outline}
    \subsection{Changing behavior}
    \begin{outline}
        \1 Phases of behavior change (transtheoretical model)
            \2 Precontemplation $\to$ Contemplation $\to$ Determination $\to$ Action $\to$ Relapse $\to$ Maintenance $\to$ Precontemplation (and so on)
            \2 A person may exit or reenter at any point within the cycle. 
                \3 Precontemplation: people don't intend to take any action in the forseeable future (6 months); they are unaware of their negative behavior. 
                \3 Contemplation: people start to realize that their behavior is problematic and intend to take action soon (within 6 months). 
                \3 Determination/preparation: people are ready to make a change within the next month. They take small steps towards the positive outcome. 
                \3 Action: people have recently changed their behavior within the past 6 months and intend to continue. 
                \3 Maintenance: people have changed their behavior for over 6 months, and intend to continue. They work on preventing \textbf{relapse} into earlier stages. 
                \3 Termination: people have no desire to return to the problematic behavior and are completely sure they will not relapse. This stage is rarely reached. 
        \1 Behavioral interventions
            \2 Ways to influence people to change their behavior
            \2 Example 1: a study had participants log their transit habits for a week; at the end of the week, a reseracher discussed the results with them and suggested specific improvements that could be made. 
            \2 Example 2: commuter challenge - people commute to work using different modes of transportation in a mini experiment and see which one is the fastest. It could be a personal project or just something that someone does and publishes it to the media. Usually, the bicycle wins and using a helicopter doesn't win. 
            \2 Example 3: social sanctions: in Bogota, the government had people dress up as traffic cones and playfully harrass people who parked illegally, causing embarassement as a deterrent for illegal parking. The key here is that it is done seemingly jokingly and not in a hostile way, but it still achieves its intended result. 
            \2 Other strategies: you could make things into a game with incentives
            \2 Tactical urbanism: visually change a space, like a parking spot or new bike lane (with things like paint, furniture, plants, etc.), to signify that it is now serving a different purpose. 
    \end{outline}
    \subsection{Equity}
    \begin{outline}
        \1 In transportation planning, certain vulnerable groups are often overlooked and therefore are not included in transit opportunities/accessibility. These groups include: children, women, elderly people, disabled people, and BIPOC communities. 
        \1 People who are disadvantaged by lacking transit accessibility miss out on economic opportunities and are subject to a perpetuation of poverty. 
    \end{outline}
    \subsection{Health and environment}
    \begin{outline}
        \1 
    \end{outline}
\end{document}