\documentclass{article}
\title{Notes Section 2} % title
\author{John Yang}
\usepackage{amsmath}
\usepackage[margin=1in, letterpaper]{geometry}
\usepackage{outlines}
\setcounter{section}{1} % chapter number minus 1
\usepackage{mathtools}
\DeclarePairedDelimiter\set\{\}
\usepackage{hyperref}
\hypersetup{
	colorlinks=true,
	linkcolor=blue,
	filecolor=magenta,      
	urlcolor=cyan,
}
\usepackage{tocloft}
\renewcommand\cftsecfont{\normalfont}
\renewcommand\cftsecpagefont{\normalfont}
\renewcommand{\cftsecleader}{\cftdotfill{\cftsecdotsep}}
\renewcommand\cftsecdotsep{\cftdot}
\renewcommand\cftsubsecdotsep{\cftdot}

\begin{document}
    \maketitle
    \tableofcontents
    \section{11.550x - Leveraging Urban Mobility Disruptions to Create Better Cities}
    \subsection{Course Introduction}
    \begin{outline}
        \1 Overarching themes: transportation as it applies to Cities 
            \2 Status quo: personal cars are used for most travel/trips
            \2 Future goal: for cities where zero-e and multimodal travel is not only available and accessible but the new normal for inhabitants. 

    \end{outline}
    \subsection{The Status Quo}
    \begin{outline}
        \1 Over 100 years ago, cars were gradually made the standard choice for transportation (mobility). It was reasonable because of what came before it: horses; cities were less dense. 
        \1 Now, cities and towns have been designed around the automobile, leading to sprawling urban and suburban areas, traffic congestion (which economically harms the city). 
        \1 20-50M people globally are disabled nonfatally by car crashes yearly; 1.35 died from car crashes in 2018. 
        \1 Traffic accidents are the leading cause of death of children 15-24 in the US 
        \1 93\% of traffic deaths occur in low/middle income countries
        \1 Over half of all road death victims are pedestrians, cyclists, and motorcyclists. 
        \1 Statistically, public transit is THE safest way to travel. 
        \1 Environmental: Cars cause poor air quality, athsma, and global warming
        \1 In the average household, transportation is the second largest budget item 

    \end{outline}
    \subsection{Consequences of auto-centric models}
    \begin{outline}
        \1 Our transportation system disadvantages many classes of people, including racial minorities, people in home countries, etc. 
        \1 Our transportation system is extractive bc it depends on carbon fuels and imported vehicles; which disadvantages many. 
        \1 In most places, you are \textit{forced} to have a car to partake in typical activities; some places, you can survive without a car, but barely.
            \2 \textit{Therefore}, poorer people are forced to buy secondhand or less reliable cars, which puts them at a further economic disadvantage, despite their preexisting disadvantage. 
            
    \end{outline}
\end{document}