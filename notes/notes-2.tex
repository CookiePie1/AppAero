\documentclass{article}
\title{Notes Section 2} % title
\author{John Yang}
\usepackage{amsmath}
\usepackage[margin=1in, letterpaper]{geometry}
\usepackage{outlines}
\setcounter{section}{1} % chapter number minus 1
\usepackage{mathtools}
\DeclarePairedDelimiter\set\{\}
\usepackage{hyperref}
\hypersetup{
	colorlinks=true,
	linkcolor=blue,
	filecolor=magenta,      
	urlcolor=cyan,
}
\usepackage{tocloft}
\renewcommand\cftsecfont{\normalfont}
\renewcommand\cftsecpagefont{\normalfont}
\renewcommand{\cftsecleader}{\cftdotfill{\cftsecdotsep}}
\renewcommand\cftsecdotsep{\cftdot}
\renewcommand\cftsubsecdotsep{\cftdot}

\begin{document}
    \maketitle
    \tableofcontents
    \section{11.550x - Leveraging Urban Mobility Disruptions to Create Better Cities}
    \subsection{Course Introduction}
    \begin{outline}
        \1 Overarching themes: transportation as it applies to Cities 
            \2 Status quo: personal cars are used for most travel/trips
            \2 Future goal: for cities where zero-e and multimodal travel is not only available and accessible but the new normal for inhabitants. 

    \end{outline}
    \subsection{The Status Quo}
    \begin{outline}
        \1 Over 100 years ago, cars were gradually made the standard choice for transportation (mobility). It was reasonable because of what came before it: horses; cities were less dense. 
        \1 Now, cities and towns have been designed around the automobile, leading to sprawling urban and suburban areas, traffic congestion (which economically harms the city). 
        \1 20-50M people globally are disabled nonfatally by car crashes yearly; 1.35 died from car crashes in 2018. 
        \1 Traffic accidents are the leading cause of death of children 15-24 in the US 
        \1 93\% of traffic deaths occur in low/middle income countries
        \1 Over half of all road death victims are pedestrians, cyclists, and motorcyclists. 
        \1 Statistically, public transit is THE safest way to travel. 
        \1 Environmental: Cars cause poor air quality, athsma, and global warming
        \1 In the average household, transportation is the second largest budget item 

    \end{outline}
    \subsection{Consequences of auto-centric models}
    \begin{outline}
        \1 Our transportation system disadvantages many classes of people, including racial minorities, people in home countries, etc. 
        \1 Our transportation system is extractive bc it depends on carbon fuels and imported vehicles; which disadvantages many. 
        \1 In most places, you are \textit{forced} to have a car to partake in typical activities; some places, you can survive without a car, but barely.
            \2 \textit{Therefore}, poorer people are forced to buy secondhand or less reliable cars, which puts them at a further economic disadvantage, despite their preexisting disadvantage. 
    \end{outline}
    \subsection{Improved mobility methods}
    \begin{outline}
        \1 Car based:
            \2 Car-sharing
                \3 New concepts where users can rent cars for short periods of time to get from one place to another; then, another user takes the car to the next place. 
                \3 Bike-sharing concepts like citibike already exist in widespread use; car-sharing is different because there is no need for a designated bike rack or storage location. 
            \2 Ride-sharing 
                \3 Companies like Lyft and Uber where drivers use their own vehicles to drive customers around. 
                \3 Multiple people can use the service at once (two passengers in the same car, pooling together though their destinations may not be the same)
        \1 Micromobility
            \2 Includes services that share bikes, scooters, electric motorcycles, etc. 
            \2 A downside is that small devices can be left in public spaces where they aren't returned to the proper parking location, becoming an eyesore/danger/negative thing. 
        \1 Microtransit
            \2 Like ride sharing but with more passengers (6-12). Kind of like a mini bus service with many vehicles traveling around. 
        \1 E-commerce and E-retail
            \2 Increases reliance on delivery services. Strictly speaking, this can be much more efficient than going out yourself to buy things, especially if you depend on cars to do so. Because packages and products can be shipped in bulk, you spread your emissions between a large number of people. 

    \end{outline}
    \subsection{Ridesourcing and transit}
    \begin{outline}
        \1 Ridesourcing was found to have its major market in low-income urban neighborhoods first rather than downtown or affluent areas. 
        \1 This is economically unfavorable for those who use it, as pricing for ridesourcing services is highly fluctuous and it tends to be much more expensive than traditional transit methods, which are already inaccessible to low-income neighborhoods. 
        \1 However, because these communities are relying increasingly on ridesharing, government-sponsored transit will turn its focus away from these already underserved communities, further disparaging them. 
        \1 Low-income neighborhoods tend to have much higher populations of black, latino, and other racial minorities. 

    \end{outline}
Overall, people tend to be blind to the harms that cars and personal car-centric models cause to individuals as well as society. 
\end{document}