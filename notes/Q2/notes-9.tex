\documentclass{article}
\title{Notes Section 9} % title
\author{John Yang}
\usepackage{amsmath}
\usepackage[margin=1in, letterpaper]{geometry}
\usepackage{outlines}
\setcounter{section}{8} % chapter number minus 1
\usepackage{mathtools}
\DeclarePairedDelimiter\set\{\}
\usepackage{hyperref}
\hypersetup{
	colorlinks=true,
	linkcolor=blue,
	filecolor=magenta,      
	urlcolor=cyan,
}
\usepackage{tocloft}
\renewcommand\cftsecfont{\normalfont}
\renewcommand\cftsecpagefont{\normalfont}
\renewcommand{\cftsecleader}{\cftdotfill{\cftsecdotsep}}
\renewcommand\cftsecdotsep{\cftdot}
\renewcommand\cftsubsecdotsep{\cftdot}

\begin{document}
    \maketitle
    \tableofcontents
    \section{Land Use and Zoning}
    \subsection{Influential Factors}
    \begin{outline}
        \1 Utilities
            \2 Plumbing, sewage
            \2 Electricity
        \1 Environment
            \2 Sunlight
                \3 Ex. Elevated rail lines in NYC were rejected in favor of subways because they made streets dark, even in the middle of the day. 
            \2 Rain 
                \3 Drainage and flood prevention systems
            \2 Air quality
                \3 Pollution, air purification
                \3 Place heavy industries away from areas such as residential. 
            \2 Wind
                \3 Building structures, channeling of air
        \1 Basic geography
            \2 Mountains, rivers, hills, etc. 
            \2 Freshwater sources
            \2 Soil texture
        \1 Social interactions
            \2 How do we facilitate interaction and coexistence through the built environment?
        \1 Economics
            \2 Location of businesses 
            \2 Mixed use development
            \2 Transit oriented development
        \1 Historic preservation
            \2 Are there any important buildings we need to conserve when constructing new ones?
        \1 Transportation
            \2 How can we make transportation accessible to people everywhere in the city, both by designing transit service and by designing buildings and their locations?

    \end{outline}
    \subsection{Types of Land Uses}
    \begin{outline}
        \1 Residential 
            \2 Low-density: single-family homes 
            \2 Medium-density: rowhouses and townhouses
            \2 High-density: Apartment buildings 
            \2 Mixed-use: residential combined with commercial, transit, etc. 
        \1 Commercial
            \2 Restaurants
            \2 Retail 
            \2 Offices
            \2 Warehouses \& light logistics
        \1 Agricultural
            \2 Farms, grazing fields
        \1 Public use 
            \2 Municipal buildings 
            \2 Hospitals 
            \2 Schools
            \2 Churches
        \1 Industrial
            \2 Manufacturing
            \2 Warehouses
            \2 Shipping yards
            \2 Rail yards
            \2 Heavy logistics/industry
        \1 Transportation 
            \2 Railroads, roads, highways
            \2 Airports
            \2 Train stations 
        \1 Recreational
            \2 Parks
            \2 Golf courses
            \2 Open spaces/gathering areas/squares
            \2 Athletic fields
            \2 Swimming pools etc.
    \end{outline}
    \subsection{How do we decide where to put our buildings?}
    \begin{outline}
        \1 Zoning is relatively new. Yet, in increasingly populated and fast-grawing cities, it is necessary to allow cities to grow sustainably. 
        \1 Zoning regulations are flexible to accomodate growth or new needs that arise. 
        \1 In US History:
            \2 Before zoning there was redlining, which informed developers on where to spend their money and resources developing communities. As a result of segregation, certain areas were more heavily invested in and the effects are still visible today. Most major cities in the US were and still are heavily segregated. 
        \1 Zoning regulations can encourage growth in certain areas that are seen as high-potential, and they can also be used to discourage development in certain areas for environmental reasons, historic preservation, or other reasons. 
    \end{outline}
\end{document}