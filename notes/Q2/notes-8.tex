\documentclass{article}
\title{Notes Section 8} % title
\author{John Yang}
\usepackage{amsmath}
\usepackage[margin=1in, letterpaper]{geometry}
\usepackage{outlines}
\setcounter{section}{7} % chapter number minus 1
\usepackage{mathtools}
\DeclarePairedDelimiter\set\{\}
\usepackage{hyperref}
\hypersetup{
	colorlinks=true,
	linkcolor=blue,
	filecolor=magenta,      
	urlcolor=cyan,
}
\usepackage{tocloft}
\renewcommand\cftsecfont{\normalfont}
\renewcommand\cftsecpagefont{\normalfont}
\renewcommand{\cftsecleader}{\cftdotfill{\cftsecdotsep}}
\renewcommand\cftsecdotsep{\cftdot}
\renewcommand\cftsubsecdotsep{\cftdot}

\begin{document}
    \maketitle
    \tableofcontents
    \section{Urban Housing Solutions}
    \subsection{Homelessness}
    \begin{outline}
        \1 Top causes of homelessness:
            \2 Lack of affordable housing 
            \2 Unemployment and/or low income
            \2 Poverty
            \2 Mental illness and lack of needed services
            \2 Substance abuse and the lack of needed services
            \2 For women: domestic violence
        \1 Different definitions of homelessness:
            \2 HUD: Only includes "traditionally" homeless people: those living in shelters, transitional housing, or public spaces. 
            \2 Dept of Education: includes families "doubled up" with others due to economic reasons. 
        \1 Counts and data of homeless populations are extremely difficult and often inaccurate. 
        \1 Most likely demographic to be homeless are adult, male, Black, not elderly, unaccompanied/alone, and disabled. 
    \end{outline}
    \subsection{Infrastructure and Policy}
    \begin{outline}
        \1 A lot of urban infrastructure is designed hostilly against homeless people, and in the process, is also hostile towards disabled people. For example, benches in subway stations are often replaced with leaning non-seats, or removed altogether. Spikes are placed in areas where homeless people might gather. 
        \1 Recently, there has been an increase in tents that homeless people use for shelter. 
        \1 Many systems in society prevent homeless people from recovering. For example, homeless people can't open bank accounts or credit cards because they require an address. Many jobs with larger corporations require an address to apply, limiting people to under-the-table employment or jobs at small businesses. 
        \1 Then, once they do get a job, how will they get paid? Most businesses don't pay employees in cash, and some require direct deposit or use checks. But how do you use these if you don't have a bank account? Having large amounts of cash also makes you more susceptible to crime, including theft and/or assault. 
        \1 Most remedies to homelessness currently take the form of charity. While charity is great for some people, it does nothing to change the system to benefit everyone, not just the people that benefactors have decided to help. Charity keeps oppressors in control. 
    \end{outline}
    \subsection{Types of Urban Homes}
    \begin{outline}
        \1 Density is key: urban life improves when housing density is at least medium to high. However, overly high density housing, like in Hong Kong, rids people of personal space and people feel cramped/suffocated. 
        \1 Two most common urban housing solutions are rowhomes and high-rise apartments. Other types of apartments also exist, such as walkups and courtyard apartments. 
    \end{outline}
    \subsection{Affordable Housing}
    \begin{outline}
        \1 Housing at an adequate living quality can be subsidized by the government or some other organization to make it more affordable for lower income people. 
        \1 Affordable housing is not meant to have worse living conditions. That is inhumane. 
    \end{outline}
\end{document}