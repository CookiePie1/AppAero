\documentclass{article}
\title{Notes Section 3} % title
\author{John Yang}
\usepackage{amsmath}
\usepackage[margin=1in, letterpaper]{geometry}
\usepackage{outlines}
\setcounter{section}{2} % chapter number minus 1
\usepackage{mathtools}
\DeclarePairedDelimiter\set\{\}
\usepackage{hyperref}
\hypersetup{
	colorlinks=true,
	linkcolor=blue,
	filecolor=magenta,      
	urlcolor=cyan,
}
\usepackage{tocloft}
\renewcommand\cftsecfont{\normalfont}
\renewcommand\cftsecpagefont{\normalfont}
\renewcommand{\cftsecleader}{\cftdotfill{\cftsecdotsep}}
\renewcommand\cftsecdotsep{\cftdot}
\renewcommand\cftsubsecdotsep{\cftdot}

\begin{document}
    \maketitle
    \tableofcontents
    \section{11.550x: Land Use and Urban Form}
    \subsection{Auto-centricity: past, present, and future}
    \begin{outline}
        \1 Why did cities build highways that were bad for them?
            \2 Companies like GM and AAA grouped together and started lobbying for tax-funded highways. 
            \2 GM designed a highway system that debuted at the 1939 World's Fair that featured expwys that connected cities and ran right through them, allowing for greater traffic flow and decreased congestion. 
            \2 When highways cut through cities, they require the demolition of entire communities, the victims of which are usually low-income or minorities. 
            \2 Highway designers were usually members of the auto industry, but not urban planners (who did not really exist at the time)
            
    \end{outline}
\end{document}