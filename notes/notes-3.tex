\documentclass{article}
\title{Notes Sections 3} % title
\author{John Yang}
\usepackage{amsmath}
\usepackage[margin=1in, letterpaper]{geometry}
\usepackage{outlines}
\setcounter{section}{2} % chapter number minus 1
\usepackage{mathtools}
\DeclarePairedDelimiter\set\{\}
\usepackage{hyperref}
\hypersetup{
	colorlinks=true,
	linkcolor=blue,
	filecolor=magenta,      
	urlcolor=cyan,
}
\usepackage{tocloft}
\renewcommand\cftsecfont{\normalfont}
\renewcommand\cftsecpagefont{\normalfont}
\renewcommand{\cftsecleader}{\cftdotfill{\cftsecdotsep}}
\renewcommand\cftsecdotsep{\cftdot}
\renewcommand\cftsubsecdotsep{\cftdot}

\begin{document}
    \maketitle
    \tableofcontents
    \section{11.550x: Land Use and Urban Form}
    \subsection{Auto-centricity: past, present, and future}
    \begin{outline}
        \1 Why did cities build highways that were bad for them?
            \2 Companies like GM and AAA grouped together and started lobbying for tax-funded highways. 
            \2 GM designed a highway system that debuted at the 1939 World's Fair that featured expwys that connected cities and ran right through them, allowing for greater traffic flow and decreased congestion. 
            \2 When highways cut through cities, they require the demolition of entire communities, the victims of which are usually low-income or minorities. 
            \2 Highway designers were usually members of the auto industry, but not urban planners (who did not really exist at the time)
        \1 As automobiles became the primary mode of transportation for most of US communities, developing highways and other auto infrastructure became an excuse to demolish entire neighborhoods or communities, which were overwhelmingly low-income or underprivileged. 
        \1 In Washington DC: highway building led to the slogan "No white men's roads through black men's homes" bc the highway construction was seen as targeting black neighborhoods. 
    \end{outline}
    \subsection{Land Use}
    \begin{outline}
        \1 Over half of the world's population lives in an urban area
        \1 How did cities originate?
            \2 Some postulate that cities originated as marketplaces; this is why so many cities are based around bodies of water where ports were a major source of trade. 
        \1 Today, cities exist because they help reduce the cost of mobility. This is why lower-income people are associated with inner cities, though housing costs may be more expensive. Suburban communities are typically middle to upper middle class. 
        \1 Central Business District (CBD): the "downtown" areas of cities, where much of the business is conducted. Urban planners propose that different areas of land should be positioned with distances to the CBD based on the cost of transport for those resources. 
            \2 Simply put, land use is a function of transport costs to market and land price. 
        \1 % Land Use 2
    \end{outline}
\end{document}
