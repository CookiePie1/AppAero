\documentclass{article}
\title{Notes Sections 3} % title
\author{John Yang}
\usepackage{amsmath}
\usepackage[margin=1in, letterpaper]{geometry}
\usepackage{outlines}
\setcounter{section}{2} % chapter number minus 1
\usepackage{mathtools}
\DeclarePairedDelimiter\set\{\}
\usepackage{hyperref}
\hypersetup{
	colorlinks=true,
	linkcolor=blue,
	filecolor=magenta,      
	urlcolor=cyan,
}
\usepackage{tocloft}
\renewcommand\cftsecfont{\normalfont}
\renewcommand\cftsecpagefont{\normalfont}
\renewcommand{\cftsecleader}{\cftdotfill{\cftsecdotsep}}
\renewcommand\cftsecdotsep{\cftdot}
\renewcommand\cftsubsecdotsep{\cftdot}

\begin{document}
    \maketitle
    \tableofcontents
    \section{11.550x: Land Use and Urban Form}
    \subsection{Auto-centricity: past, present, and future}
    \begin{outline}
        \1 Why did cities build highways that were bad for them?
            \2 Companies like GM and AAA grouped together and started lobbying for tax-funded highways. 
            \2 GM designed a highway system that debuted at the 1939 World's Fair that featured expwys that connected cities and ran right through them, allowing for greater traffic flow and decreased congestion. 
            \2 When highways cut through cities, they require the demolition of entire communities, the victims of which are usually low-income or minorities. 
            \2 Highway designers were usually members of the auto industry, but not urban planners (who did not really exist at the time)
        \1 As automobiles became the primary mode of transportation for most of US communities, developing highways and other auto infrastructure became an excuse to demolish entire neighborhoods or communities, which were overwhelmingly low-income or underprivileged. 
        \1 In Washington DC: highway building led to the slogan "No white men's roads through black men's homes" bc the highway construction was seen as targeting black neighborhoods. 
    \end{outline}
    \subsection{Land Use}
    \begin{outline}
        \1 Over half of the world's population lives in an urban area
        \1 How did cities originate?
            \2 Some postulate that cities originated as marketplaces; this is why so many cities are based around bodies of water where ports were a major source of trade. 
        \1 Today, cities exist because they help reduce the cost of mobility. This is why lower-income people are associated with inner cities, though housing costs may be more expensive. Suburban communities are typically middle to upper middle class. 
        \1 Central Business District (CBD): the "downtown" areas of cities, where much of the business is conducted. Urban planners propose that different areas of land should be positioned with distances to the CBD based on the cost of transport for those resources. 
            \2 Simply put, land use is a function of transport costs to market and land price. 
        \1 Certain urban forces pull people together and others push people apart. Finding the right density requires a city to find the balance between these two forces. The resulting form shows a prioritization of factors. 
            \2 Pull: Increased density = less geographical separation = lower transportation cost = more frequent human-to-human interaction = better productivity and improvement of technology and ideas. 
            \2 Push: High-density urban areas become congested and polluted. The richer people come, the more they demand their own personal space, and the more resources they have access to, allowing them to personally reduce their own transportation inefficiencies, which separates them from the urban area. 
        \1 This is why sprawled-out urban areas become unstable, as geographical sepraation results in increased personal vehicle usage, which causes a positive feedback cycle, further spreading the urban area. 
    \end{outline}
    \subsection{Accessiblity}
    \begin{outline}
        \1 Accessiblity: the extent to which the land use and transportation systems enable groups of individuals to reach activities and destinations. 
        \1 Thus, accessiblity depends on a number of factors: efficiency, speed, capacity, fuel consumption, location of access points, equality of access, etc. 
        \1 Another important accessibility factor is information and communication. If public transit is to be reliable and efficient, it is important that the pax can know where the vehicles are and whether they can count on the information they are receiving being accurate. 
        \1 Cars and other personal vehicles have a very high consumption of fuel per passengers moved, while something like walking has a very low factor. Obviously, it's not practical for people to walk everywhere, which is why mass transit is so important. 
        \1 How do we design transportation systems so that even those who can afford using a car will use public transit instead? What is the opportunity cost and benefit for these passengers?
        \1 Our infrastructure is designed for cars. 
        \1 Accessiblity is the main goal of transportation; in order to increase accessibility, we can increase mobility OR increase proximity. 
        \1 Different systems of accessibility interact with each other:    
            \2 Land use systems control \textit{where} things of interest/purpose are, geographically
            \2 Mobility systems influence the desirability of certain places, controlling \textit{how} we get places. 
        \1 If the built environment influcences mobility in creating accessibility, this property can be planned and designed to improve mobility and accessibility in a given community. 
        \1 The bulit environment influences trip costs by discouraging or encouraging certain modes based on the relative appeal of the mode. For example, how much does it cost to get from A to B by driving vs. taking public transit? How long does it take? Which is more convenient?
        \1 How do we observe the effects of the built environment on communities? 
            \2 It is difficult to apply traditional scientific experimental methods to such studies because of the complex interactions between other factors such as social context. 
            \2 Many studies have been done within small contexts but it is difficult to generalize all of their conclusions into a simple, coherent idea. 
    \end{outline}
\end{document}
